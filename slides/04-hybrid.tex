\documentclass{beamer}


% Beamer settings
\usecolortheme{rose}
\beamertemplatenavigationsymbolsempty
\setbeamertemplate{footline}[frame number]

% Packages
\usepackage{amsmath}

\usepackage{pgfplots}
\usepgfplotslibrary{fillbetween}

\usepackage{minted}
\usepackage[T1]{fontenc} % Required by minted to ensure dollar signs are produced instead of pound (sterling) signs

\usepackage{multicol}

\usepackage{booktabs}


\title{OpenMP for Computational Scientists}
\subtitle{4: Combining MPI and OpenMP}

\begin{document}

\frame{\titlepage}

%-------------------------------------------------------------------------------
\begin{frame}
\frametitle{Outline}
\begin{itemize}
  \item Non-uniform Memory Access
  \item Thread affinity in OpenMP
  \item Scaling details
  \item Combining MPI with OpenMP
\end{itemize}
\end{frame}

%-------------------------------------------------------------------------------
\begin{frame}
\frametitle{NUMA}
\end{frame}

%-------------------------------------------------------------------------------
\begin{frame}
\frametitle{OpenMP proc bind}
\end{frame}

%-------------------------------------------------------------------------------
\begin{frame}
\frametitle{Why combine MPI+OpenMP}
Include scaling results
\end{frame}

%-------------------------------------------------------------------------------
\begin{frame}[fragile]
\frametitle{MPI programs}
What happens when you run an MPI program?
\begin{minted}{bash}
mpirun -np 16 ./a.out
\end{minted}

\begin{itemize}
  \item 16 processes are spawned on one (or more) nodes according to the hostname list file given by the queueing system.
  \item There is no reason why these processes have to be serial:
  \begin{itemize}
    \item Each MPI rank could spawn OpenMP threads and run in parallel.
    \item Each MPI rank could use a GPU.
  \end{itemize}
\end{itemize}

  MPIInitthread

\end{frame}

%-------------------------------------------------------------------------------

\end{document}
