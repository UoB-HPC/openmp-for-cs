\documentclass{beamer}


% Beamer settings
\usecolortheme{rose}
\beamertemplatenavigationsymbolsempty
\setbeamertemplate{footline}[frame number]

% Packages
\usepackage{amsmath}

\usepackage{pgfplots}
\usepgfplotslibrary{fillbetween}

\usepackage{minted}
\usepackage[T1]{fontenc} % Required by minted to ensure dollar signs are produced instead of pound (sterling) signs

\usepackage{multicol}

\usepackage{booktabs}


\author{Tom Deakin}
\title{OpenMP for Computational Scientists}

\begin{document}

\frame{\titlepage}

%-------------------------------------------------------------------------------
\section{Outline}
\begin{frame}
\frametitle{Course Outline}
\begin{itemize}
  \item OpenMP overview (fork join, spmd, work sharing) (vec add)
  \item Data sharing clauses and reduction (Pi program, critical, single atomics)
  \item Matrix multiplication optimisation
  \item Jacobi
  \item SIMD clauses
  \item Tasks
  \item NUMA
  \item GPU programming
  \item MPI interop (MPI\_Thread\_init, who can send messages, hybrid scaling pros and cons, thread pinning)
\end{itemize}
\end{frame}

%-------------------------------------------------------------------------------
\begin{frame}
Tim's slides about growth/history of OpenMP
\end{frame}

%-------------------------------------------------------------------------------
\begin{frame}[fragile]
\frametitle{Building with OpenMP}

\begin{minted}{bash}
gfortran *.f90 -fopenmp # GNU
ifort *.f90 -qopenmp    # Intel
ftn *.f90               # Cray (on by default)
pgf90 *.f90 -mp         # PGI
\end{minted}

To also use the API calls within the code, use the library:
\begin{minted}{fortran}
USE omp_lib
\end{minted}

\begin{alertblock}{Note}
No need to include the library if only using the compiler directives.
The library only gets you the API calls.
\end{alertblock}


\end{frame}

%-------------------------------------------------------------------------------

\end{document}
