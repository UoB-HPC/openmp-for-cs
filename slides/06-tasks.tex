\documentclass{beamer}


% Beamer settings
\usecolortheme{rose}
\beamertemplatenavigationsymbolsempty
\setbeamertemplate{footline}[frame number]

% Packages
\usepackage{amsmath}

\usepackage{pgfplots}
\usepgfplotslibrary{fillbetween}

\usepackage{minted}
\usepackage[T1]{fontenc} % Required by minted to ensure dollar signs are produced instead of pound (sterling) signs

\usepackage{multicol}

\usepackage{booktabs}


\title{OpenMP for Computational Scientists}
\subtitle{6: Tasking and Tools}

\begin{document}

\frame{\titlepage}

%-------------------------------------------------------------------------------
\begin{frame}
\frametitle{Outline}
\begin{itemize}
  \item Poor mans tasking: sections
  \item The single and master constructs
  \item Tasking in OpenMP
  \item Tools
\end{itemize}
\end{frame}

%-------------------------------------------------------------------------------
\begin{frame}[fragile]
\frametitle{The need for tasks}
\begin{itemize}
  \item What if your code doesn't follow a standard parallel loop pattern?
  \item OpenMP needs to know the loop count at runtime.
  \item Recursive and tree/graph based algorithms inconvinient to program with parallel loops.
  \item Tasking allows you to package work and data into units (tasks) and have them scheduled in parallel.
\end{itemize}

\begin{minted}[frame=single]{fortran}
p => head
do while(associated(p))
  call process(p)
  p => p%next
end do
\end{minted}
\end{frame}
%-------------------------------------------------------------------------------
\section{Sections}
\begin{frame}[fragile]
\frametitle{Sections}
\begin{itemize}
  \item Sections give you a way to assign different work to different threads.
  \item Useful for a producer/consumer pattern, but not really recursive algorithms.
  \item If fewer sections than threads, threads sit idle.
  \item If more sections than threads, sections assigned by implementation.
\end{itemize}

\begin{minted}[frame=single]{fortran}
!$omp parallel sections
    !$omp section
    call work1()

    !$omp section
    call work2()

!$omp end parallel sections
\end{minted}
\end{frame}

%-------------------------------------------------------------------------------
\section{Single and master}
\begin{frame}[fragile]
\frametitle{The single construct}
\begin{itemize}
  \item Often necessary for only one thread to do some work in a parallel region.
  \item Wrapping code with the \mintinline{fortran}|single| constructs means only one thread in the parallel region will execute that code.
  \item Not defined which thread will actually execute the block.
  \item There is an implicit barrier for this construct (can avoid with \mintinline{fortran}|nowait| clause).
\end{itemize}

\begin{minted}[frame=single]{fortran}
!$omp parallel
  call do_work()
  !$omp single
  call exchange_halos()
  !$omp end single
!$omp end parallel
\end{minted}
\end{frame}

%-------------------------------------------------------------------------------
\begin{frame}[fragile]
\frametitle{The master construct}
\begin{itemize}
  \item Similar to \mintinline{fortran}|single| construct, except guarenteed the master thread will execute the block.
  \item There is \emph{no} implicit barrier for this construct!
  \item Useful for cases where synchronisation isn't required:
    \begin{itemize}
      \item Printing out messages to the screen.
      \item Generating tasks!
    \end{itemize}
\end{itemize}

\begin{minted}[frame=single]{fortran}
!$omp parallel
  call do_work()
  if (conv .eq. .true.)
    !$omp master
    print *, "Converged!"
    !$omp end master
    exit
  end if
!$omp end parallel
\end{minted}
\end{frame}

%-------------------------------------------------------------------------------
\section{Tasks}
\begin{frame}[fragile]
\frametitle{Tasks}
\begin{itemize}
  \item OpenMP 3.0 introduced real tasking.
  \item Later versions refined and added features.
\end{itemize}

\begin{columns}
\begin{column}{0.5\textwidth}
\begin{minted}[frame=single,fontsize=\small]{fortran}
!$omp parallel
  !$omp master
    !$omp task
    call do_task(x,y,z)
    !$omp end task

    !$omp task
    call do_task(i,j,k)
    !$omp end task
  !$omp end master
!$omp end parallel
\end{minted}
\end{column}

\begin{column}{0.5\textwidth}
\begin{itemize}
  \item Use master thread to generate tasks for all threads to do.
  \item Idle threads take tasks off the task queue.
  \item Barrier at the \mintinline{fortran}|end parallel| ensures all tasks complete.
\end{itemize}
\end{column}
\end{columns}

\end{frame}

%-------------------------------------------------------------------------------
\begin{frame}
\frametitle{Trees}
Useful to think about tasks organised as a tree (or graph).

\begin{columns}
\begin{column}{0.5\textwidth}
\begin{tikzpicture}[every node/.style = {shape=rectangle, draw}]
  \node (it) {Initial Task};
  \node (t1) [below=of it, xshift=-1cm] {Task A} edge [<-] (it);
  \node (t2) [below=of it, xshift=1cm] {Task B} edge [<-] (it);
  \node (t3) [below=of t1, xshift=-1cm] {Task C} edge [<-] (t1);
  \node (t4) [below=of t1, xshift=1cm] {Task D} edge [<-] (t1);

\end{tikzpicture}
\end{column}

\begin{column}{0.5\textwidth}
\begin{itemize}
  \item The Initial Task is the \emph{parent} of Task A.
  \item Task A is the \emph{child} of the Initial Task.
  \item Task A and Task B are \emph{siblings}.
\end{itemize}
\end{column}
\end{columns}
\end{frame}

%-------------------------------------------------------------------------------
\begin{frame}
\frametitle{Data sharing clauses}
\begin{itemize}
  \item Can use these three data sharing clauses from \mintinline{fortran}|parallel| regions on the \mintinline{fortran}|task| construct.
  \item The defintions are really the same, but applied to tasks, not threads.
  \item Need to think carefully about the appropriate clauses for each variable.
  \item \mintinline{fortran}|shared(x)|: There is one copy of the \mintinline{fortran}|x| variable, shared between the current and child tasks.
  \item \mintinline{fortran}|private(x)|: Each task gets its own local \mintinline{fortran}|x| variable. It is not initialised.
  \item \mintinline{fortran}|firstprivate(x)|: Each task gets its own local \mintinline{fortran}|x| variable. It is initilised to the value of the original variable in the encountering task region.
\end{itemize}
\end{frame}

%-------------------------------------------------------------------------------
\begin{frame}
\frametitle{Default data sharing}
\begin{itemize}
  \item Tasks and parallel regions have different default data sharing rules.
  \item By default, data is \emph{shared} for \mintinline{fortran}|parallel| regions.
  \item By default, data is \emph{firstprivate} for \mintinline{fortran}|task| constructs.
  \item Unless, it's shared by the enclosing (the outer) region.
  \item Using \mintinline{fortran}|default(none)| is recommended here especially.
\end{itemize}
\end{frame}

%-------------------------------------------------------------------------------
\begin{frame}
\frametitle{Task completion}
Two places where tasks synchronise:
\begin{enumerate}
  \item At thread barriers:
    \begin{itemize}
      \item Implicit barriers (like at end of \mintinline{fortran}|parallel| regions).
      \item Explicit \mintinline{fortran}|barrier| constructs.
    \end{itemize}
  \item At the \mintinline{fortran}|taskwait| construct:
    \begin{itemize}
      \item Waits on all child tasks of the current task before continuing.
      \item Only applies to tasks that the current task generated.
      \item Everything in OpenMP is defined as task, so here the ``outer'' task is the one that called the first \mintinline{fortran}|task| construct.
    \end{itemize}
\end{enumerate}
\end{frame}

%-------------------------------------------------------------------------------
\begin{frame}
\frametitle{Back to the tree}

\begin{columns}
\begin{column}{0.5\textwidth}
\begin{tikzpicture}[every node/.style = {shape=rectangle, draw}]
  \node (it) {Initial Task};
  \node (t1) [below=of it, xshift=-1cm] {Task A} edge [<-] (it);
  \node (t2) [below=of it, xshift=1cm] {Task B} edge [<-] (it);
  \node (t3) [below=of t1, xshift=-1cm] {Task C} edge [<-] (t1);
  \node (t4) [below=of t1, xshift=1cm] {Task D} edge [<-] (t1);
\end{tikzpicture}
\end{column}

\begin{column}{0.5\textwidth}
\begin{itemize}
  \item A \mintinline{fortran}|taskwait| in the Initial Task will wait for Task A and B to finish. Task A finishes when Tasks C and D are also finished.
  \item A \mintinline{fortran}|taskwait| in Task A will wait for Tasks C and D to finish.
  \item A \mintinline{fortran}|taskwait| in Task B will do nothing (no children).
\end{itemize}
\end{column}
\end{columns}
\end{frame}

%-------------------------------------------------------------------------------
\begin{frame}[fragile]
\frametitle{Fibonacci}
\begin{columns}
\begin{column}{0.5\textwidth}
\begin{minted}[breaklines,frame=single,linenos,fontsize=\small]{fortran}
recursive integer function fib(n) result(res)
  integer :: n, i, j
  if (n .lt. 2) then
    res = n
  else
    !$omp task shared(i)
    i = fib(n-1)
    !$omp end task
    !$omp task shared(j)
    j = fib(n-2)
    !$omp end task
    !$omp taskwait
    res = i+j
  end if
end function
\end{minted}
\end{column}

\begin{column}{0.5\textwidth}
\begin{itemize}
  \item Routine called from a parallel region by one of the threads.
  \item This first task \mintinline{fortran}|fib(40)| creates two tasks: \mintinline{fortran}|fib(39)| and \mintinline{fortran}|fib(38)|.
  \item \mintinline{fortran}|i| and \mintinline{fortran}|j| must be \mintinline{fortran}|shared| so that the results are retained by the calling task.
  \item The \mintinline{fortran}|taskwait| construct ensures that the child tasks are finished before summing their return values.
\end{itemize}
\end{column}

\end{columns}
NB: there are better ways to calculate Fibonacci numbers\dots
\end{frame}

%-------------------------------------------------------------------------------
\begin{frame}
\frametitle{Task dependencies}
\begin{itemize}
  \item OpenMP gives you the ability to define an ordering of tasks based on input and output data dependancies.
  \item In other words, can make tasks wait for other sibling tasks to finish before starting.
  \item Need to specify input and output dependencies.
  \item Only checks for previously generated siblings with a dependency specified.
  \item \mintinline{fortran}|depend(in: A)|: task depends on siblings with \mintinline{fortran}|out| or \mintinline{fortran}|inout| dependancy \mintinline{fortran}|A|.
  \item \mintinline{fortran}|depend(out: A)|: task depends on siblings with any dependancy on \mintinline{fortran}|A|.
  \item \mintinline{fortran}|depend(inout: A)|: same as \mintinline{fortran}|depend(out: A)|.
\end{itemize}
\end{frame}

%-------------------------------------------------------------------------------
\begin{frame}[fragile]
\frametitle{Dependency example}
\begin{columns}
\begin{column}{0.7\textwidth}
\begin{minted}[frame=single,linenos,fontsize=\small]{fortran}
!$omp task depend(out: x)
x = 1.0
!$omp end task

!$omp task depend(in: x) depend(out: y)
y = f(x)
!$omp end task

!$omp task depend(in: x) depend(out: z)
z = g(x)
!$omp end task

!$omp task depend(in: y, z)
print *, y+z
!$omp end task
\end{minted}
\end{column}

\begin{column}{0.3\textwidth}
\begin{tikzpicture}[every node/.style = {shape=rectangle, draw}]
  \node (it) {$x = 1.0$};
  \node (t1) [below=of it, xshift=-1cm] {$y=f(x)$} edge [<-] (it);
  \node (t2) [below=of it, xshift=1cm] {$z=g(x)$} edge [<-] (it);
  \node (t3) [below=of t2, xshift=-1cm] {$y+z$} edge [<-] (t1) edge (t2);
\end{tikzpicture}
\end{column}
\end{columns}
\end{frame}

%-------------------------------------------------------------------------------
\begin{frame}[fragile]
\frametitle{Dependency example}
\begin{columns}
\begin{column}{0.5\textwidth}
\begin{minted}[breaklines,frame=single,linenos,fontsize=\small]{fortran}
!$omp task depend(out: x)
x = 1.0
!$omp end task

!$omp task depend(in: x) depend(out: y)
y = f(x)
!$omp end task

!$omp task depend(in: x) depend(out: z)
z = g(x)
!$omp end task

!$omp task depend(in: y, z)
print *, y+z
!$omp end task
\end{minted}
\end{column}

\begin{column}{0.5\textwidth}
\begin{itemize}
  \item Must specify first \mintinline{fortran}|out| dependency.
  \item Then generate $y=f(x)$ tasks which depends on it.
  \item Then generate $z=g(x)$ task, which depends on first but not second.
  \item Generate task which depends on middle two tasks.
  \item Note, OpenMP still sees all these tasks as \emph{siblings}.
\end{itemize}
\end{column}
\end{columns}

\end{frame}

%-------------------------------------------------------------------------------
\begin{frame}
\frametitle{Clauses and things}
untied, mergable, Priority, final, taskyield, taskloop
\end{frame}
%-------------------------------------------------------------------------------
\begin{frame}
\frametitle{Tasking advice}
\begin{itemize}
  \item Getting the correct data sharing clauses can be tricky.
  \item Don't use tasks for patterns supported by other parts of OpenMP (parallel loops).
  \item Tasking comes with overheads.
  \item The runtime is good, but can't work miracles.
  \item Best results obtained where use controls number and granularity of tasks.
\end{itemize}
\end{frame}

%-------------------------------------------------------------------------------
\begin{frame}
\frametitle{Resources}
\begin{itemize}
  \item SC'16 Tutorial from Tim Mattson and Alice Koniges: \url{https://www.nersc.gov/assets/Uploads/SC16-Programming-Irregular-Applications-with-OpenMP.pdf}
\end{itemize}
\end{frame}
%-------------------------------------------------------------------------------

\end{document}

