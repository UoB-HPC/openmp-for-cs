\documentclass{beamer}


% Beamer settings
\usecolortheme{rose}
\beamertemplatenavigationsymbolsempty
\setbeamertemplate{footline}[frame number]

% Packages
\usepackage{amsmath}

\usepackage{pgfplots}
\usepgfplotslibrary{fillbetween}

\usepackage{minted}
\usepackage[T1]{fontenc} % Required by minted to ensure dollar signs are produced instead of pound (sterling) signs

\usepackage{multicol}

\usepackage{booktabs}


\title{OpenMP for Computational Scientists}
\subtitle{6: Tasking and Tools}

\begin{document}

\frame{\titlepage}

%-------------------------------------------------------------------------------
\begin{frame}
\frametitle{Outline}
\begin{itemize}
  \item Poor mans tasking: sections
  \item The single and master constructs
  \item Tasking in OpenMP
  \item Tools
\end{itemize}
\end{frame}

%-------------------------------------------------------------------------------
\begin{frame}[fragile]
\frametitle{Tasking}
\begin{itemize}
  \item What if your code doesn't follow a standard parallel loop pattern?
  \item OpenMP needs to know the loop count at runtime.
  \item Recursive and tree/graph based algorithms inconvinient to program with parallel loops.
  \item Tasking allows you to package work and data into units (tasks) and have them scheduled in parallel.
\end{itemize}

\begin{minted}[frame=single]{fortran}
p => head
do while(associated(p))
  call process(p)
  p => p%next
end do
\end{minted}
\end{frame}
%-------------------------------------------------------------------------------
\section{Sections}
\begin{frame}
\frametitle{Sections}
\begin{itemize}
  \item What if your work isn't 
\end{itemize}
\end{frame}

%-------------------------------------------------------------------------------

\end{document}

